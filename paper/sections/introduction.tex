\section{Introduction}
\label{sec:introduction}

\subsection{Problem Statement: The Opacity Crisis in LLM Reasoning}

The remarkable success of large language models (LLMs) in complex reasoning tasks has been accompanied by a critical limitation: the fundamental opacity of their decision-making processes \cite{brown2020language,openai2023gpt4}. While these models demonstrate impressive capabilities across mathematical reasoning, logical inference, and problem-solving, their ``black box'' nature presents significant barriers to deployment in high-stakes applications requiring explainable artificial intelligence.

This opacity manifests in several critical ways:
\begin{itemize}
    \item \textbf{Trust and Verification}: Users cannot verify the correctness of reasoning steps, limiting trust in model outputs for critical decisions
    \item \textbf{Error Diagnosis}: When models produce incorrect results, the absence of visible reasoning makes error diagnosis and correction extremely difficult
    \item \textbf{Regulatory Compliance}: Many domains (healthcare, finance, legal) require explainable AI systems that can justify their decisions
    \item \textbf{Educational Applications}: Learning systems need to demonstrate reasoning processes to effectively teach problem-solving strategies
\end{itemize}

Current approaches to address this opacity either sacrifice significant performance for interpretability or require substantial architectural modifications that limit practical deployment.

\subsection{Related Work}

\subsubsection{Chain-of-Thought Prompting}
Chain-of-thought prompting \cite{wei2022chain} represents the foundational approach to eliciting reasoning from language models by encouraging step-by-step problem decomposition. Extensions include self-consistency \cite{wang2022self}, tree-of-thoughts \cite{yao2023tree}, and least-to-most prompting \cite{zhou2022least}. However, these approaches only expose the final reasoning chain, not the internal deliberation process.

\subsubsection{Constitutional AI and Self-Improvement}
Constitutional AI \cite{anthropic2022constitutional} demonstrates how models can be trained to critique and improve their own outputs through self-supervision. Recent work on weak-to-strong generalization \cite{burns2023weak} shows that smaller models can be trained using outputs from larger models to achieve improved performance. Our approach extends these concepts by creating recursive self-improvement loops specifically for reasoning transparency.

\subsubsection{Scratchpad and Intermediate Computation}
Scratchpad methods \cite{nye2021show} allow models to perform intermediate computations in a visible workspace. While related to our approach, these methods focus primarily on computational steps rather than the complete reasoning process including problem analysis, strategy selection, and verification.

\subsection{Our Approach: Transparent Thought Processes}

We introduce a novel methodology that makes the complete internal reasoning process explicit through structured \texttt{<thinking>} tags. Our approach differs from previous work in several key aspects:

\begin{enumerate}
    \item \textbf{Complete Process Transparency}: Unlike CoT prompting which shows only the final reasoning chain, our models expose their entire thought process including problem analysis, strategy consideration, intermediate steps, and self-verification.
    
    \item \textbf{Self-Improvement Training Loops}: We implement recursive training where models generate reasoning examples, evaluate their quality, and use improved examples for subsequent training iterations.
    
    \item \textbf{Dual-Model Architecture}: We train complementary models - \supra{} optimized for transparent reasoning and \zennano{} optimized for efficient instruction following - that together address different deployment scenarios.
    
    \item \textbf{Production-Ready Implementation}: Our models are optimized for practical deployment with multiple format support (GGUF, MLX, PyTorch) and quantization strategies.
\end{enumerate}

\subsection{Models Overview}

We introduce two complementary models built upon the \qwen{} 2B parameter architecture:

\begin{itemize}
    \item \textbf{\supra{}}: A reasoning-focused model that employs explicit \texttt{<thinking>} tags to expose complete internal reasoning processes before generating final answers. Optimized for transparency and complex problem-solving scenarios.
    
    \item \textbf{\zennano{}}: An efficient model optimized for direct instruction following while maintaining reasoning capabilities. Designed for production deployment scenarios requiring fast, reliable responses.
\end{itemize}

Both models leverage Low-Rank Adaptation (\lora{}) fine-tuning \cite{hu2021lora} for parameter-efficient training and are optimized using Apple's \mlx{} framework for efficient inference on Apple Silicon hardware.

\subsection{Contributions}

This work makes the following key contributions to transparent AI reasoning:

\begin{enumerate}
    \item \textbf{Novel Training Methodology for Transparent Reasoning}: We introduce the first systematic approach to training language models that explicitly expose their complete internal reasoning process through structured \texttt{<thinking>} tags, achieving transparency without sacrificing performance.
    
    \item \textbf{Self-Improvement Dataset Creation}: We develop a recursive self-improvement methodology where models generate, evaluate, and refine their own reasoning examples, leading to progressively enhanced reasoning capabilities across training iterations.
    
    \item \textbf{Multi-Format Deployment Pipeline}: We provide comprehensive deployment support including quantized GGUF formats for llama.cpp, optimized MLX implementations for Apple Silicon, and standard PyTorch models, enabling broad accessibility across hardware platforms.
    
    \item \textbf{Open-Source Release of Models and Training Data}: We release both \supra{} and \zennano{} models along with complete training datasets (18 structured files), training scripts, and evaluation frameworks under open-source licenses to enable reproducible research and community development.
    
    \item \textbf{Empirical Validation of Transparency-Performance Trade-offs}: We provide comprehensive benchmarks demonstrating that transparent reasoning can improve performance (22.1\% improvement in reasoning correctness) rather than degrading it, challenging the assumption that interpretability requires performance sacrifices.
\end{enumerate}

\subsection{Paper Organization}

The remainder of this paper is structured as follows: Section \ref{sec:related_work} provides comprehensive review of related work in chain-of-thought reasoning and transparent AI. Section \ref{sec:methodology} describes our training methodology, self-improvement loops, and evaluation framework. Section \ref{sec:architecture} details the technical architecture and implementation. Section \ref{sec:training} presents our training procedures and optimization strategies. Section \ref{sec:evaluation} provides experimental results and comparative analysis. Section \ref{sec:applications} discusses deployment considerations and practical applications. Section \ref{sec:discussion} analyzes implications and limitations, and Section \ref{sec:conclusion} concludes with future research directions.